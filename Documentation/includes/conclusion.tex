\chapter{Zusammenfassung und Fazit}

Die Interaktion von AR-Objekten und echten Objekten gestaltet sich schwierig. Die Performanz der Vuforia Objekt Recognition ist gut, jedoch ist das Vuforia SDK unausgereift. Das Fehlen einer präzisen Bounding Box erschwert die Berechnung der Wegfindung, da die Größe des Objekts im Spielfeld nicht klar ersichtlich ist. Zusätzlich ist Unity sehr konfigurationslastig, was auch bei kleinen Änderungen in der Konfiguration große Auswirkungen auf die Spielszene haben kann.  Mehrere Leute können nicht gleichzeitig in einer Szene arbeiten, da eine Szene aus einer Datei besteht, die sich schwierig mergen lässt. Dies erschwerte die Teamarbeit per Versionsverwaltung. Der Objekterkennungsansatz ist unserem Projekt ist sehr experimentell. Es war uns nicht möglich einen ähnlichen Ansatz für die Nutzung der Objekterkennung in einer AR Anwendung zu finden. Dies streut den Verdacht, dass AR zwar gut dafür ist AR-Objekte in die reale Welt zu platzieren, jedoch der Use-Case den wir für unser Projekt gewählt haben momentan nicht möglich ist. Dieses Projekt hatte ein großes Forschungsdelta, da wir uns gut in Unity einarbeiten konnten und viel Zeit in die Recherche und die verschiedenen Ansätze der Objekterkennung investiert haben. 


Als Ausblick könnte man in RunnAr dynamische Ziele einbauen, die durch ImageTargets repräsentiert werden. Zusätzlich könnte das vorbereite Skript für die schießende Elemente eingebaut werden, dessen Projektilen der Algorithmus zusätzlich ausweichen müsste. Dies könnte auch den Spielfluss verbessern. Der Kamerawinkel könnte angepasst werden, um die Vogelperspektive in eine isometrische Ansicht zu ändern. So hätte man ein besseren Ansicht auf die Spielumgebung. Dazu müsste man jedoch den Versuch unternehmen, die Spielfigur auf der gleichen Ebene wie die gesetzten Objekte zu platzieren. Als Ansatz würde sich hier eine Ground Detection anbieten, die erweitert werden müsste um nicht nur Objekte zu platzieren, sondern auch die Spielfigur auf dem Tisch weitgehend realistisch in Bewegung darstellt.